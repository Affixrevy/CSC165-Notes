\documentclass[12pt, letterpaper, twoside]{article}
\usepackage[utf8]{inputenc}
\usepackage{amsfonts}
\usepackage{amsmath}
\usepackage{amssymb}
\usepackage{ifthen}
\usepackage{fancyhdr}
\usepackage{tikz}
\usepackage{csc}


\oddsidemargin=.2in
\evensidemargin=.2in
\textwidth=6in
\topmargin=0in
\textheight=9.0in
\parskip=.07in
\parindent=0in
\pagestyle{fancy}



\begin{document}

\setheader{CSC165}{Lecture 12}{Arthur Gao}{}{}{}
\section{Algorithm Analysis}
Algorithm analysis has two components:
\begin{enumerate}
    \item Correctness:
    \begin{itemize}
        \item why does my program work?
        \item Does it produce the corre t results for valid inputs
        \item \emph{left to CSC236}
    \end{itemize}
    \item Complexity - aka "efficiency" 
    \begin{itemize}
    
        \item How much resources does a program use
        \begin{itemize}
            \item time
            \item space (memory)
            \item bandwidth (communication)
        \end{itemize}
        \item It is not cow concise / elegant the code itself is. 
    \end{itemize}
\end{enumerate}
Reviewing past ideas (csc108)

Say we have program:

We measures resources relative to a size measure of the input
\begin{itemize}
    \item If the input is a string, we might use the length
    \item If the input is a natural number, we might us it or maybe $log_2$ of it.
\end{itemize}
We'll assume the size is a natural number and what we measure is a non-negative number 
therefore the measure is some function: $\N \to \R^{\ge 0}$

Define:\\
    For $f, g: \N \to \R^{\ge 0}$, $f$ absolutely dominates $g$ if:
    $$
        \forall n \in \N, g(n) \le f(n)
    $$

Given that, we can define $h, j: N \to \R^{\ge 0} $ by $h(n) = 2n + 3, j(n) = \frac{n}{2} + 5$
Examining this, we note that if we say $100h$
\begin{align*}
    j(n) &= \frac{n}{2} + 5\\
    &\le 200n + 5\\
    &\le 200n + 300\\
    &= 100h(n)
\end{align*}
therefore, $n \ge 0$

Formalizing this, we can say that:

Define:\\
h dominates j up to a control factor if:
$$
\exists c \in \R^+, \forall n \in \N, j(n) \le ch(n)
$$
\begin{aproof}
    Let $c = 4 \in \R^+$. Let $n \in \N$\\
    Then,
    \begin{align*}
        h(n) &= 2n + 3\\
        &\le 2n + 20\\
        &= c(\frac{n}{2} + 5)\\
        &= cj(n)
    \end{align*}
\end{aproof}

Example:
\begin{itemize}
    \item Define $k: \N \to \R^{\ge 0}$ by $k(n) = n^2$
    \item Define $h: \N \to \R^{\ge 0}$ by $h(n) = 2n + 3$
\end{itemize}
We note that it is impossible to make it so that either $h$ dominates $k$ or $k$
dominates $g$ even up to a control factor, therefore we have a third situation:

Define:\\
For $f, g: \N \to \R^{\ge 0}$, $f$ eventually dominates $g$ up to a control factor if:
    $$
        \exists c, n_0 \in \R^+, \forall n \in \N, n \ge n_0 \to g(n) \le cf(n)
    $$
This is also written as $g \in \bigO(f)$

Formally proving the given example that $h \in \bigO(k)$ would yield:
\begin{aproof}
    Let $c = 5 \N R^+$\\
    Let $n_0 = 1 \in \R^+$\\
    \begin{align*}
        h(n) &= 2n+3\\
        &\le 2n + 3n &\text{since }n \ge 1\\
        &= 5n\\ 
        &\le 5n^2 \\
        &= cn^2 &\text{since }n \ge 1\\
        &= ck(n) &\text{since }c = 5
    \end{align*}
\end{aproof}

We can also say that theo opposite is of the above statement is :
$$k \not\in \bigO(h) \! : \! \forall c, n_0 \in \R^+, \exists n \in \N, n \ge n_0 \land k(n) > ch(n)$$

\end {document}
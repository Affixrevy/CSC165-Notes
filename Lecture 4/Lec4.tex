\documentclass[12pt, letterpaper, twoside]{article}
\usepackage[utf8]{inputenc}
\usepackage{amsfonts}
\usepackage{amsmath}
\usepackage{amssymb}
\usepackage{ifthen}
\usepackage{fancyhdr}
\usepackage{tikz}
\usepackage{csc}

\oddsidemargin=.2in
\evensidemargin=.2in
\textwidth=6in
\topmargin=0in
\textheight=9.0in
\parskip=.07in
\parindent=0in
\pagestyle{fancy}



\begin{document}

\setheader{CSC165}{Lecture 4}{Arthur Gao}{}{}{}

\section{Predicates}

\subsection{Some Rules about Predicates}
Let $S$ be the set of non-empty strings over alphabet {a, b, c} \\
\hspace*{10mm} Define binary predicate $P$ with domain $S \times S$ by $P(x, y)$: \\
\hspace*{10mm}"x and y have the same first character"\\
Another option for this definition without announcing the domain in advance:\\
\hspace*{10mm} Define $P(x, y)$ by/as: "x and y have the same first character", where $x, y \in S$
\\
\\
Both of the above statements are valid. You should not, however, put quantification in the meaning of predicate. eg.
\\
\hspace*{10mm}$x, y \in S$ for each  $P(x, y): \dots$
\\
$P(x, y):$ "x and y have \dots where $x, y \in S$ is also invalid as is assumes x, and y are in S.

\subsection{Example}
$$
\forall x \in X, \forall y \in S, P(x, y)
$$
This could be translated to:\\
\hspace*{10mm} "For each x in S and for each y in S, x and y have the same first character.\\
However, To simplify it and make it more natural english:\\
\hspace*{10mm} "For all pair of string over S have the same first character"
\\
\subsection{Grouping Quantifiers}
Note that variables with the same quantifier can be grouped together so:
$$
\forall x \in S, \forall y \in S, P(x, y) \equiv \forall x, y \in S, P(x, y)
$$
\\
\subsection{Example}
Say we wanted to say that there are a pair on non-equal strings to start with the same character, we can combine propositions with and:
$$
\exists x, y \in S, x \neq y \land P(x, y)
$$
Which is true: eg. $x = aaa, y = baab"$
\\
\subsection{Example}
Say we wanted to say \textit{if} a pair of strings are different, they start with the same first letter
$$
\exists x, y \in S, x \neq y \implies P(x, y)
$$
\\
\subsection{Example}
Given the statement:\\
\hspace*{10mm} In each pair of distinct strings, the strings start with the same character
$$
\forall x, y \in S, x \neq y \implies P(x, y)
$$
This statement is \textbf{FALSE}\\
\hspace*{10mm} Consider counter-example x = aaa, y = baa\\
\\
Note that to say that a statement is false is the same as saying that the negation is true so:
\begin{align*}
	\lnot \big(\forall x, y \in S, x \neq y \implies &P(x, y)\big)\\
	&\equiv\\
	\exists x, y \in S, x \neq y \land \lnot &P(x, y)
\end{align*}
Where in the negation, the counter-example of the original statement is an example for the negation.

\subsection{Examining Commas}
Consider the question:
$$
\forall x, y \in S, x \neq y, P(x, y)
$$
what does the comma mean?\\
Some possibilities are:
\begin{itemize}
	\item and
	\item implication
	\item such that
\end{itemize}
The convention is that:
\begin{itemize}
	\item After universal, comma \textit{often} means implication ($\implies$)
	\item After existential, comma \textit{often} means and ($\land$)
\end{itemize}
However, \textbf{DO NOT RELY ON COMMAS}. Just use precise language.
\section{Examine Divisibility}
\bbox{Definition:\\
$d$ devides $n$ (also written as $d|m$)\\
if $n = dk$ for some $k \in \Z$}
Note that in definitions, "if" almost certainly means "iff"
\end {document}
\documentclass[12pt, letterpaper, twoside]{article}
\usepackage[utf8]{inputenc}
\usepackage{amsfonts}
\usepackage{amsmath}
\usepackage{amssymb}
\usepackage{ifthen}
\usepackage{fancyhdr}
\usepackage{tikz}
\usepackage{csc}
\usepackage{listings}
\usepackage{xcolor}

\definecolor{codegreen}{rgb}{0,0.6,0}
\definecolor{codegray}{rgb}{0.5,0.5,0.5}
\definecolor{codepurple}{rgb}{0.58,0,0.82}
\definecolor{backcolour}{rgb}{0.95,0.95,0.92}

\lstdefinestyle{mystyle}{
    backgroundcolor=\color{backcolour},   
    commentstyle=\color{codegreen},
    keywordstyle=\color{magenta},
    numberstyle=\tiny\color{codegray},
    stringstyle=\color{codepurple},
    basicstyle=\ttfamily\footnotesize,
    breakatwhitespace=false,         
    breaklines=true,                 
    captionpos=b,                    
    keepspaces=true,                 
    numbers=left,                    
    numbersep=5pt,                  
    showspaces=false,                
    showstringspaces=false,
    showtabs=false,                  
    tabsize=2
}

\lstset{style=mystyle}

\oddsidemargin=.2in
\evensidemargin=.2in
\textwidth=6in
\topmargin=0in
\textheight=9.0in
\parskip=.07in
\parindent=0in
\pagestyle{fancy}



\begin{document}

\setheader{CSC165}{Lecture 13}{Arthur Gao}{}{}{}
\section{Example from last time}
Let $f, g, \N \to \R^{\ge 0}$\\
Prove that if $f \in \bigO{g}$ then, $g \in \bigO(f)$\\

\begin{aproof}
    Assume $f \in \bigO{g}$\\
    $\exists c, n_0 \in \R^+, \forall n \in \N, n \ge n_0 \implies f(n) \le cg(n)$\\
    \\ 
    Let $n_0' = n_0 \in \R^+$\\
    Let $c' = \frac{1}{c} \in \R^+$ since $c \in \R^+$\\
    Let $n \in \N$ Assume $n \ge n_0' = n_0$\\ 
    \\
    Then
    \begin{align*}
        f(n) &\le cg(n)\\
        \frac{1}{c}f(n) \le g(n)\\
        g(n) \ge c'f(n)
    \end{align*}
    Therefore,
    $$
    \exists c', n+0' \in \R^+, \forall n \in \N, n \ge n_0' \implies g(n) \ge c'f(n)
    $$
\end{aproof}

\section{Algorithm Analysis}

    We have the math tools of: $\bigO, \Omega, \Theta$\\
    We use these tools to achieve the goal of a "simple" $f$ so that $\Theta(f)$ "represents the particular resource complexity.
\\
\\
    We will focus on runtime complexity:\\
    For Algorithm $A$ and input $x$, let $RT_A(x)$ b the time to run algorithm $A$ on input $x$\\

    Say we have sorting algorithm: \code{def sort(l): } \hspace{10mm} $RT_{sort}: Lists \to \R^{\ge 0}$
    \begin{itemize}
        \item Given the definition of $\bigO, \Omega, \Theta$, they cannot take lists as input
        \item We therefore take the measure the input. ie, length of the list - \code{len(l)} $ \in \N$
        \item We also must output the list in some way.
        \begin{itemize}
            \item Worst case - WC \code{sort(n)} = max
            \item Best case - BC \code{sort(n)} = min
            \item Average Case - AC \code{sort(n)} = average
        \end{itemize}
    \end{itemize}
    
    Define Today: $x \in \N$

    \begin{lstlisting}[language=Python]
def A(n: int) -> int: #Assume n >= 0
    r = 0
    for i in range(10):
        for j in range(n * n):
            r = r + j
    for i in range (n//2):
        for j in range(i * i):
            r = r - j
    return r
    \end{lstlisting}
    Determine $RT_A(n)$

    note a step is a piece of code tht takes fixed/consistent amoutn of time to interpredtate the input.
    \hspace*{10mm}Then m steps take between $min(c)m$ and $max(c)m$ time, ie $\theta(m)$

    We analyze the above program:

    \begin{itemize}
        \item Loop lines 3-4
        \begin{itemize}
            \item Body: 1 step
            \item Iterations: $n^2$
            \item Total: $n^2 \cdot 1 = n^2$ steps
        \end{itemize}
    \end{itemize}
    \begin{itemize}
        \item Loop lines 2-4
        \begin{itemize}
            \item Body: $n^2$ step
            \item Iterations: $10$
            \item Total: $10n^2$ steps
        \end{itemize}
    \end{itemize}
    \begin{itemize}
        \item Loop lines 6-7
        \begin{itemize}
            \item Body: $i^2$ steps
        \end{itemize}
    \end{itemize}
    \begin{itemize}
        \item Loop lines 5-7
        \begin{itemize}
            \item Body: $i^2$ step
            \item Iterations: $0, 1, 2, \dots, \floor{\frac{n}{2}} - 1$
            \item Total: $0^2 + 1^2 + 2^2 + \dots + \big(\floor{\frac{n}{2}} - 1\big)^2$ steps
            \begin{itemize}
                \item $= \sum_{i = 0}^{\floor{\frac{n}{2}} - 1} i^2 \in \theta(n^3)$
            \end{itemize}
        \end{itemize}
    \end{itemize}
    Therefore the algorithm $\in \theta(1 + 10n^2 + n^3 + 4) \in \theta(n^3)$
\end {document}
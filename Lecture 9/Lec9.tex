\documentclass[12pt, letterpaper, twoside]{article}
\usepackage[utf8]{inputenc}
\usepackage{amsfonts}
\usepackage{amsmath}
\usepackage{amssymb}
\usepackage{ifthen}
\usepackage{fancyhdr}
\usepackage{tikz}
\usepackage{csc}

\oddsidemargin=.2in
\evensidemargin=.2in
\textwidth=6in
\topmargin=0in
\textheight=9.0in
\parskip=.07in
\parindent=0in
\pagestyle{fancy}



\begin{document}

\setheader{CSC165}{Lecture 9}{Arthur Gao}{}{}{}
First in person lecture Yay!!!


\section{Continuation on Proof Techniques}
Some proof techniques:
\begin{itemize}
    \item \strong{Direct Proof:} let follow structure
    \begin{itemize}
        \item Know something $\forall, \exists, \land, \lor, \implies, \leftrightarrow$
    \end{itemize}
    \item \strong{Indirect Proof:} when the proofs rely on equivalences
    \begin{itemize}
        \item $P \implies Q \equiv \lnot Q \implies \lnot P$
        \item $P \land R \implies Q \equiv Q \implies P \land R$
        \item $P \implies (Q \implies R) \equiv (P \land Q) \implies R$
        
        Examining $P \implies (Q \implies R)$, we look at the proof structure:
        \begin{proof}
            Assume P
            (WTS $Q \implies R$)\\
            Assume Q \dots \\
            Therefore: R
        \end{proof}
        \\
        \\
        Examining proof structure of $(P \land Q) \implies R$:

        \begin{proof}
            Assume P and Assume Q\\
            (WTS R)\\
            Therefore: R
        \end{proof}

        We can note that similar proofs can be an indication of equivalences
    \end{itemize}
\end{itemize}

\subsection{Proof by Contradiction}
Proof by contradiction relies on:
$$
P \equiv \lnot P \implies \False
$$
We therefore can note that:
\begin{align*}
    P \implies Q &\equiv P \implies (\lnot Q \implies \False)\\
    &\equiv (P \land \lnot Q) \implies \False
\end{align*}
Esamining the proof structures of  :\\
\begin{proof}
    Assume P\\
    Therefore: Q
\end{proof}

Esamining the proof structures of  :\\

\strong{Example:} prove $\forall n, d k \in \Z^+, n = kd \implies k \le n \land d \le n$
\begin{proof}
    Let $n, d, k \in \Z^+$,\\
    Assume $n = kd$\\
    Assume for contradiction that\footnote{This is used when doing proof by contradiction to show proof method used} 
    $k > n \lor d > n$

    \strong{Case $k > n$}\\
    Since:
    \begin{align*}
        k &\ge 1 &(\because d \in \Z^+)\\
        dk &\ge k\\
        n &\ge k > n\\
        n &> n &\text{A contradiction}
    \end{align*}

    \strong{Case $d > n$}\\
    \emph{Similar as d and k are interchangeable}
\end{proof}

\emph{Note that it is quite easy to find \strong{MANY} contradiction however it is
very easy to find a \strong{false contradiction}}

\strong{Example: } Prove $P \implies Q \equiv \lnot Q \implies \lnot Q \equiv P \implies (\lnot Q \implies \False) \equiv (P \land \lnot Q) \implies \False$

Examining the proof structure of $\lnot Q \implies \lnot P$
\begin{proof}
    Assume $\lnot Q$\\
    Therefore $\lnot P$
\end{proof}
\\
\\
Examining the proof structure of $(P \land \lnot Q) \implies \False$
\begin{proof}
    Assume $P$\\
    Assume $\lnot Q$\\
    Therefore: \False
\end{proof}

We not that a proof of the contrapositive can be transformed easily to produce the proof
of the contradiction.

We observe this behavior as above when the first proof is the contrapositive which
states that $\lnot P$ is implied while in the second proof we have \emph{assumed} $P$ 

\subsection{Induction Proofs}
Express: "There are infinitely many primes"

We know that $Prime(3) \land \exists m \in \N, m > 3 \land Prime(m)$\\
\hspace*{10mm} eg. $Prime(11)$

We also know that $$\forall n \in \N, Prime(n) \implies \exists m \in \N, m > n \land Prime(m) \land \exists n \in \N, Prime(n) $$
\end {document}
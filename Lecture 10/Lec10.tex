\documentclass[12pt, letterpaper, twoside]{article}
\usepackage[utf8]{inputenc}
\usepackage{amsfonts}
\usepackage{amsmath}
\usepackage{amssymb}
\usepackage{ifthen}
\usepackage{fancyhdr}
\usepackage{tikz}
\usepackage{csc}

\oddsidemargin=.2in
\evensidemargin=.2in
\textwidth=6in
\topmargin=0in
\textheight=9.0in
\parskip=.07in
\parindent=0in
\pagestyle{fancy}



\begin{document}

\setheader{CSC165}{Lecture 9}{Arthur Gao}{}{}{}
\section{Reviewing Contradiction Proofs}

Proving $P \implies Q$ By Contradiction. We know that 
$$
\lnot (P \implies Q) \equiv p \land \lnot Q
$$
Then our proof could follow the structure:

\begin{aproofs}
    Assume $P \land \lnot Q$\\
    \dots\\
    Therefore, Contradict.
\end{aproofs}

Alternatively we can start it directly:

\begin{aproofs}
    Assume P\\
    WTS Q by Contradict.\\
    Assume $\lnot P$\\
    \dots\\
    Therefore, Contradict.
\end{aproofs}

Recall the proof of $\forall n, k, d \in \Z^+, n = kd \implies k \le n \land d \le n$

\begin{aproof}
    Let $n, k d \in \Z^+. Assume n = dk$\\
    Assume for contra, $k > n \lor d > n$\\
    Case $k > n$
    \hspace*{5mm} \emph{Continuing solving this, we note that this case is unused.}
    \begin{align*}
        n &= kd\\
        \therefore k &= \frac{n}{d}\\
        &\le n &\therefore d \ge 1
    \end{align*}
\end{aproof}

\section{Induction Proofs}
Consider the question:
\begin{quote}
    For which $n \in \N$ is $n + 2 < n^{2-1}$?
\end{quote}
We can examine the the table
\begin{center}
    \begin{tabular}{c c c c}
        \hline
        $n$ &$n + 2$ &$2^{n - 1}$ & $n + 2 < n^{2-1}$\\
        \hline
        $0$ &$2$ &$\frac{1}{2}$ &\False\\
        $1$ &$3$ &$1$ &\False\\
        $2$ &$4$ &$2$ &\False\\
        $3$ &$5$ &$4$ &\False\\
        $4$ &$6$ &$8$ &\True\\
        $5$ &$7$ &$16$ &\True\\
    \end{tabular}
\end{center}

We come up with the conjecture:
$$
\forall n \in \N, n \ge 4 \implies n + 2 M 2^{n - 1}
$$
We can examine this conjecture by extending the table:
\begin{align*}
    7 &> 16\\
    8 = 7 + 1 &< 7 + 16 &\text{Since } 2^{n - 1} \text{ always increases by at least }16\\
    &< 16 + 16\\
    &< 32
\end{align*}
We also note:\\
Let $P(n)$ be $n + 2 < 2^{n - 1}$, then $(n + 1) + 2 = (n + 2) + 1$\\
Therefore, $2^{(n + 1) - 1} = 2 \cdot 2^{n - 1} = 2^{n - 1} \cdot 2^{n - 1}$

We notice that we can prove $P(4) \implies P(5) \land P(5) \implies P(6) \land P(6) \implies P(7) \dots$

\subsection*{Example}
let $c \in \N$\\
Suppose $P(c) \land \forall n \in \N, n \ge c \implies \big(P(n) \implies P(n + 1)\big)$\\
Then $\forall n \in \N, n \ge c \implies P(n)$\\
\begin{aproofs}
    \base Prove $P(c)$\\
    \istep Let $n \in \N$\\
    Assume $n \ge c$ and $P(n)$ -- this is by the induction hypothesis, \emph{explicitely state it in proofs}\\
    \dots\\
    Therefore $P(n + 1)$
\end{aproofs}

Proving the statement from above, 
\begin{align*}
    P{n}&: n + 2 < 2^{n - 1}\\
    Q(n)&: n \ge 4 \implies n + 2 < 2^{n - 1} \qquad \forall n \in \N, Q(n)
\end{align*}
\newpage
\begin{aproof}
    Try induction from 0.\\
    \base $0 \ge 4 \implies \dots \qquad$ Vacuously True!\\
    \istep Let $n \in \N$ Assume $n \ge 4 \implies n + 2 < 2^{n - 1}$\\
    \hspace*{10mm}\emph{WTS: $n + 1 \ge 4 \implies n + 1 + 2 < 2^{n + 1 - 1}$}
    Assume $n + 1 \ge 4$\\
    Then $n \ge 3$\\
    
    \case{$n = 3$} WTS $n + 1 + 2 < 2$\\
    \case{$n \ge 4$} WTS $n + 2 < n^{n - 1}$ By the induction hypothesis

\end{aproof}
\end {document}
\documentclass[12pt, letterpaper, twoside]{article}
\usepackage[utf8]{inputenc}
\usepackage{amsfonts}
\usepackage{amsmath}
\usepackage{amssymb}
\usepackage{ifthen}
\usepackage{fancyhdr}
\usepackage{tikz}
\usepackage{csc}

\oddsidemargin=.2in
\evensidemargin=.2in
\textwidth=6in
\topmargin=0in
\textheight=9.0in
\parskip=.07in
\parindent=0in
\pagestyle{fancy}



\begin{document}

\setheader{CSC165}{Lecture 6}{Arthur Gao}{}{}{}

\subsection{Example Proofs}
Example:\\
Prove that every integer $x$ that divides $x + 5$ also divides $5$. Formalized that is:
$$
\forall x \in \Z, (x | x + 5) \implies (x | 5)
$$
Remember that $a|b : \exists k \in \Z, b = ka$. That means that the initial definition expanded is:
$$
\forall x \in \Z, (\exists k \in \Z, x + 5 = kx) \implies (\exists k \in \Z, 5 = kx)
$$
\hspace*{5mm}\textit{Note that the first $k$ is not the same as the second $k$ in the above equation}
\\
\\
This means we have a universal that contains an implication, which in turn means we follow the structure of universal proofs shown below:
\begin{proof}
	Let $x \in \Z$\\
	\hspace*{17mm}Assume $(\exists k \in \Z, x + 5 = kx)$\\
	\hspace*{17mm}\dots\\
	\hspace*{17mm}Therefore $(\exists k \in \Z, 5 = kx)$
\end{proof}\\
\\
For the proof, we will try examples in the universal as rough work to work out the universal proof:
\begin{proof}
	Let $x \in \Z$\\
	\hspace*{17mm}Assume $(\exists k_1 \in \Z, x + 5 = k_1x)$\\
	\hspace*{17mm}Let $ = k_1 - 1$\\
	\hspace*{17mm}W.T.S. $k_2 \in \Z \equiv 5 = k_2x$\\
	\hspace*{17mm}Then, $k_2 \in \Z$ since $k_1 \in \Z$\\
	\hspace*{17mm}Also,\\
	\begin{align*}
		k_2x &= (k_1 - 1)x\\
		&= k_1x - x\\
		&= x + 5 - x\\
		&= 5
	\end{align*}
\end{proof}
\\
Exercise:\\
Prove that $\forall d, x \in \Z, x | x + d \impliesls x | d$
\end {document}
\documentclass[12pt, letterpaper, twoside]{article}
\usepackage[utf8]{inputenc}
\usepackage{amsfonts}
\usepackage{amsmath}
\usepackage{amssymb}
\usepackage{ifthen}
\usepackage{fancyhdr}
\usepackage{tikz}
\usepackage{csc}

\oddsidemargin=.2in
\evensidemargin=.2in
\textwidth=6in
\topmargin=0in
\textheight=9.0in
\parskip=.07in
\parindent=0in
\pagestyle{fancy}



\begin{document}

\setheader{CSC165}{Lecture 5}{Arthur Gao}{}{}{}

\section{Proofs}
What is a proof:\\
\bbox{In CSC165, we consider proofs as a "convincing argument"}
A proof is
\begin{itemize}
	\item Relative to an audience. Audiences can include:
	\begin{itemize}
		\item Yourself (checking yourself - convince yourself)
		\item Peers
	\end{itemize}
	\item A way to save someone time by the level of detail and organization of the exploration
\end{itemize}

When writing a proof, you must remember \strong{why}. It is a structured presentation, often in a different
order than the initial exploration of the material.
\\
\\
\subsection{Some general forms}
The Direct Proof of an existential.
\begin{proof}
	Prove: $\exists x \in D, P(x)$\\
	The \strong{Structure} often follows:\\
	Let $x_0 = $ \\
	Check / show / prove\\
	\dots \\
	Show that $x_0 \in D$ (for the chosen $x_0$)\\
	Check/ show / prove (again)\\
	Show that $P(0)$
\end{proof}
\\
The Direct proof of universal:
\begin{proof}
	Prove: $\forall x \in D, P(x)$\\
	The \strong{Structure} often follows:\\
	Let $x \in D$ - Think in terms of individual values that doesn't' depend on which particular value is chosen.
	\\
	Then prove $P(x)$
\end{proof}
\\
\\
Notice that for the existential, we introduce an individual value and write a proof \strong{for that value.}
\\
\newpage
To prove:
\begin{itemize}
	\item $\exists x \in D, P(x)$: We choose the $x$
	\item $\forall x \in D, P(x)$: Someone else chooses $x \in D$. Our proof must work without knowing which $x$.
\end{itemize}

\subsection{Further Exploration}
"Particular" vs "Arbitrary"\\
Explore "Prove $P \implies Q$\\
\begin{proof}
	Structure follows:\\
	Assume: $P$ is \True\\
	Show that 
\end{proof}
\subsection{Sample Proofs}
Prove that if a natural number is larger than $20$ then its square minus $165$ is at least four\\
\hspace*{10mm} First, we make this statement precise $\forall n \in \N, n > 20 \implies n^2 - 165 \ge 4$\\
\hspace*{10mm} Second, We do some rough work\dots\\
Try a goal:
\begin{align*}
	n^2 - 165 &\ge 4\\
	n^2 &\ge 169\\
	n &\ge 13
\end{align*}
We might have found an "equivalent condition" above.\\
Once we have completed the rough work, we can begin with the formal proof.\\
\begin{proof}
	Let $n \in \N$.\\
	Suppose $n > 20$\\
	Then, $n \ge 13$, so $n^2 \ge 169$\\
	Hence $n^2 - 165 \ge169-165$\\
	Thus $n^2 - 165 \ge 4$
\end{proof}
\\
Another way to prove the same thing\\
\begin{proof}
	Let $n \in \N$\\
	Assume $n > 20$\\
	Then: 
	\begin{align*}
		n^2 - 165 &> 400 - 165\\
		n^2 - 165 &> 235\\
		n^2 - 165 &\ge 4
	\end{align*}
\end{proof}

\end {document}
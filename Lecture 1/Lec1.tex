\documentclass[12pt, letterpaper, twoside]{article}
\usepackage[utf8]{inputenc}
\usepackage{amsfonts}
\usepackage{amsmath}
\usepackage{amssymb}
\usepackage{ifthen}
\usepackage{fancyhdr}
\usepackage{tikz}


\oddsidemargin=.2in
\evensidemargin=.2in
\textwidth=6in
\topmargin=0in
\textheight=9.0in
\parskip=.07in
\parindent=0in
\pagestyle{fancy}

\newcommand{\setheader}[6]{
	\lhead{{\sc #1}\\{\sc #2} ({\small \it \today})}
	\rhead{
		{\bf #3} 
		\ifthenelse{\equal{#4}{}}{}{(#4)}\\
		{\bf #5} 
		\ifthenelse{\equal{#6}{}}{}{(#6)}%
	}
}

\newcommand{\bbox}[1]{
    \noindent\fbox{%
        \parbox{\textwidth}{%
            #1
        }%
    }
}

%%%
% Set up some shortcut commands
%%%
\newcommand{\R}{\mathbb{R}}
\newcommand{\N}{\mathbb{N}}
\newcommand{\Z}{\mathbb{Z}}
\newcommand{\Proj}{\mathrm{proj}}
\newcommand{\Perp}{\mathrm{perp}}
\newcommand{\proj}{\mathrm{proj}}
\newcommand{\Span}{\mathrm{span}}
\newcommand{\Null}{\mathrm{null}}
\newcommand{\Rank}{\mathrm{rank}}
\newcommand{\mat}[1]{\begin{pmatrix}#1\end{pmatrix}}

\begin{document}

\setheader{CSC165}{Lecture 1}{Arthur Gao}{}{}{}

What is a set?

\boxed{\text{A set is a collection of "elements", "numbers", "items"}}

\begin{itemize}
	\item Must be distinct elements (no repeats)
	\item Example would be $\{3, 1, 2, 3\} = \{2, 1, 3\} = \{1, 2, 3\}$
	\item $\{ \dots \}$ is a notation used to REFER to a set
\end{itemize}

Take the set $\{3, 1, 2, 3\}$ we can denote the "size" or "cardinality" as
$$|\{3, 1, 2, 3\}| = |\{1, 2, 3\}| = 3$$
Suppose $x, y \in \R$ then $1 \le |\{x, y, x+y\}| \le 3$ since $x, y$ are not necessarily unique values.
Therefore, if $x = 0, y = 0$ then size of the set is 1. If $x = 0, y = 1$ then size is 2, and if $x = 1, y = 2$, size is 3.
\\
\\
EXAMPLES:
\begin{align*}
	|\{2, 3, 1\}| &= 3\\
	|\{\}| = |\emptyset| &= 0\\
	|\{\R\}| &= \infty\\
	|\{165, \text{dog}, \text{:)}\}| &= 3\\
	|\{\{2, 3, 1\}, \text{dog}, \text{:)}, 165\}| &= 4\\
	|\{\emptyset\}| &= 1\\
\end{align*}
Note that size of a finite set is always a natural number so $|s| \in \N$)
\\
\\
We use the notation $e \in S:$ to say $e$ is an ELEMENT of $S$
\\
\\
EXAMPLES:
\begin{align*}
	A &= \{\{a, b\}, c, \{d, e, f\}\}\\
	\{a, b\} &\in A\\
	c &\in A
\end{align*}
\\
\\
We use the notation $B \subseteq C:$ to say $B$ is an SUBSET of $C$ $e \in B$ then $e \in C$ where $B, C$ can be equal.
\\
We use the notation $B \subset C:$ to say $B$ is an SUBSET of $C$ $e \in B$ then $e \in C$ where $B, C$ cannot be equal.
\\
\\
EXAMPLES:
\begin{align*}
	A = \{a, b, c\}\\
	\emptyset \subset A\\
	\{a\}\\
	\{b\}\\\{c\}\\\{ab\}\\\{ac\}\\\{bc\}\\\{a, b, c\}
\end{align*}
Note: For every set $S$, $\emptyset \subseteq S$ and therefore $\emptyset \subseteq \emptyset$
\end{document}

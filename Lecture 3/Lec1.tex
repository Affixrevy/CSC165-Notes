\documentclass[12pt, letterpaper, twoside]{article}
\usepackage[utf8]{inputenc}
\usepackage{amsfonts}
\usepackage{amsmath}
\usepackage{amssymb}
\usepackage{ifthen}
\usepackage{fancyhdr}
\usepackage{tikz}


\oddsidemargin=.2in
\evensidemargin=.2in
\textwidth=6in
\topmargin=0in
\textheight=9.0in
\parskip=.07in
\parindent=0in
\pagestyle{fancy}

\newcommand{\setheader}[6]{
	\lhead{{\sc #1}\\{\sc #2} ({\small \it \today})}
	\rhead{
		{\bf #3} 
		\ifthenelse{\equal{#4}{}}{}{(#4)}\\
		{\bf #5} 
		\ifthenelse{\equal{#6}{}}{}{(#6)}%
	}
}

\newcommand{\bbox}[1]{
    \noindent\fbox{%
        \parbox{\textwidth}{%
            #1
        }%
    }
}

%%%
% Set up some shortcut commands
%%%
\newcommand{\R}{\mathbb{R}}
\newcommand{\N}{\mathbb{N}}
\newcommand{\Z}{\mathbb{Z}}
\newcommand{\Proj}{\mathrm{proj}}
\newcommand{\Perp}{\mathrm{perp}}
\newcommand{\proj}{\mathrm{proj}}
\newcommand{\Span}{\mathrm{span}}
\newcommand{\Null}{\mathrm{null}}
\newcommand{\Rank}{\mathrm{rank}}
\newcommand{\mat}[1]{\begin{pmatrix}#1\end{pmatrix}}

\begin{document}

\setheader{CSC165}{Lecture 3}{Arthur Gao}{}{}{}

\section{Propositional Logic}
Consider the following statements
\begin{itemize}
	\item "It's snowing" (here and now)
	\item "I'm wearing a hat"
\end{itemize}
\textit{These statements can either be true or false}\\
Given this, we can use \textbf{Implications} so:\\
\bbox{If it's snowing, then I'm wearing a hat - $S \implies H$. This is also a proposition}
It is important to note that the implication is solely dependent on $S$\\
\\
The truth table for implications is:
\begin{center}
	\begin{tabular}{c c c}
		\hline
		$S$ &$H$ &$S \implies H$\\
		\hline
		F &F &T\\
		F &T &T\\
		T &F &F\\
		T &T &T
	\end{tabular}
\end{center}
In the cases for when $S$ is false, these are when the implication is \textbf{vacuously true}.\\
\\
\subsection{Contrapositive}
We then consider the case with $S \implies H$ when $\lnot H$
\indent if $S$ were true then $H$ would be but $\lnot H$ so it can't be true. Therefore:
$$
S \implies H \equiv \lnot H \implies \lnot S
$$
This is the contrapositive. We can examine this in the truth table
\begin{center}
	\begin{tabular}{c c c}
		\hline
		$S$ &$H$ &$\lnot H \implies \lnot S$\\
		\hline
		T &T &T\\
		F &T &T\\
		T &F &F\\
		F &F &T
	\end{tabular}
\end{center}
\subsection{Converse}
Examine the case when $H \implies S$
\begin{center}
	\begin{tabular}{c c c}
		\hline
		$S$ &$H$ &$ H \implies S$\\
		\hline
		F &F &T\\
		F &T &F\\
		T &F &T\\
		T &T &T
	\end{tabular}
\end{center}
As the truth table above shows, the converse of the statement is not the same. Therefore,
we must be diligent in the usage of "if" in conditions.
\\
\\
However if we want both implications to be true then we can make them dependent on each other
$$
S \implies H \land H \implies S \equiv S \Leftrightarrow H
$$
\section{Predicates}
\bbox{Recall: a predicate is a function with codomain \{T, F\}}
Define: $P: Times$
by $P(t)$ is "it's snowing at time $t$"
We can quantify:\\
Let $T = Times$\\
$\forall x \in T, P(x)$
means: "For each/all/every/any time $x$, it's snowing at time $x$. which is quite verbose. That statement is equivalent to "It's always snowing" 
This can be interpreted as in a set of all times $T = \{t_1, t_2, t_3\dots\}$ then P($t_1$) $\land$ P($t_2$) $\land$ P($t_3$) $\land \dots$
\\
\\
We can also use existential statements where: $\exists x \in T, P(x)$ which means "There is/exists a time $x$ such that it's snowing at time $x$"
which can be shortened to "It snows some time at least once" so given the above set $T$, P($t_1$) $\lor$ P($t_2$) $\lor$ P($t_3$) $\lor \dots$
\subsection{Binary Predicate}
Let $D$ be the set of string over alphabet \{a, b, c\} \\
\hspace*{10mm} Note that in CS theory, ""(double quote) is omitted when referring to strings so $\epsilon$ is used to refer to the empty string.
Therefore, the set $D = $
Define $P : D \times D \to \{T, F\}$ by\\
\indent $P(x, y): x$ and $y$ have the same first character.\\
We can consider the case where $P(\epsilon, \epsilon)$, we realize that this predicate actually has two interpretations:
\begin{itemize}
	\item $\land: x, y$ have first characters and they are the same.
	\item $\implies:$ if $x, y$ have first characters they are the same. 
\end{itemize}
\subsection{Examining Quantifiers}
\begin{itemize}
	\item $\exists x \in D, \exists y \in D, P(x, y)$: True (since we can pick any value - even the same)
	\item $\forall x \in D, \forall y \in D, P(x, y)$: False (since we can come up with counterexamples eg. x = baa, y = cab)
	\item $\forall x \in D, \exists y \in D, P(x, y)$: Depends on the empty string interpretation
	\begin{itemize}
		\item And interpretation: False since the predicate is false for any empty string
		\item Implication interpretation: True since the empty string case is vacuously true.
	\end{itemize}
	\item $\exists x \in D, \forall y \in D, P(x, y)$: This is essentially the same as above and has interpretations.
\end{itemize}


\end {document}
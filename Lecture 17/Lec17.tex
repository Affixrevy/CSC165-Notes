\documentclass[12pt, letterpaper, twoside]{article}
\usepackage[utf8]{inputenc}
\usepackage{amsfonts}
\usepackage{amsmath}
\usepackage{amssymb}
\usepackage{ifthen}
\usepackage{fancyhdr}
\usepackage{tikz}
\usepackage{csc}
\usepackage{listings}
\usepackage{xcolor}
\usepackage{pgfplots}

\definecolor{codegreen}{rgb}{0,0.6,0}
\definecolor{codegray}{rgb}{0.5,0.5,0.5}
\definecolor{codepurple}{rgb}{0.58,0,0.82}
\definecolor{backcolour}{rgb}{0.95,0.95,0.92}

\lstdefinestyle{mystyle}{
    backgroundcolor=\color{backcolour},   
    commentstyle=\color{codegreen},
    keywordstyle=\color{magenta},
    numberstyle=\tiny\color{codegray},
    stringstyle=\color{codepurple},
    basicstyle=\ttfamily\footnotesize,
    breakatwhitespace=false,         
    breaklines=true,                 
    captionpos=b,                    
    keepspaces=true,                 
    numbers=left,                    
    numbersep=5pt,                  
    showspaces=false,                
    showstringspaces=false,
    showtabs=false,                  
    tabsize=2
}

\lstset{style=mystyle}

\oddsidemargin=.2in
\evensidemargin=.2in
\textwidth=6in
\topmargin=0in
\textheight=9.0in
\parskip=.07in
\parindent=0in
\pagestyle{fancy}



\begin{document}

\setheader{CSC165}{Lecture 17}{Arthur Gao}{}{}{}

\section{Worst-Case, Best-Case, Average-Case, Runtime}

Recall: for algoritm A with domain of input $I$, we choose a natural number size (measure for input)

Then for $n \in \N$ let $\mathcal{I}_n = \{z in \mathcal{I}: z\text{ has size n}\}$

and $WC_A(n) = max \{RT_A(z), z \in \mathcal{I}_n\}$

and $BC_A(n) = min \{RT_A(z), z \in \mathcal{I}_n\}$

So the upper bound and lower bounds can be obtained by:
\begin{align*}
    WC \in \bigO(f) \text{ iff } \exists c, n_0, \in \R^{\ge 0}, \forall n \in \N, n \ge n_0 &\implies \forall z \in \mathcal{I}_n, RT(z) \le c f(n)\\
    WC \in \Omega(f) \text{ iff } \exists c, n_0, \in \R^{\ge 0}, \forall n \in \N, n \ge n_0 &\implies \exists z \in \mathcal{I}_n, RT(z) \ge c f(n)\\
    BC \in \bigO(f) \text{ iff } \exists c, n_0, \in \R^{\ge 0}, \forall n \in \N, n \ge n_0 &\implies \exists z \in \mathcal{I}_n, RT(z) \le c f(n)\\
    BC \in \Omega(f) \text{ iff } \exists c, n_0, \in \R^{\ge 0}, \forall n \in \N, n \ge n_0 &\implies \forall z \in \mathcal{I}_n, RT(z) \ge c f(n)
\end{align*}
Given the code 
\begin{lstlisting}[language=Python]
def is_in(x: int l: List) -> bool:
    for item in l:
        if x == item:
            return True
    return False
\end{lstlisting}

Notice that for all conditions, the number of iterations is less than or equal to $n$

\end {document}
\documentclass[12pt, letterpaper, twoside]{article}
\usepackage[utf8]{inputenc}
\usepackage{amsfonts}
\usepackage{amsmath}
\usepackage{amssymb}
\usepackage{ifthen}
\usepackage{fancyhdr}
\usepackage{tikz}

\oddsidemargin=.2in
\evensidemargin=.2in
\textwidth=6in
\topmargin=0in
\textheight=9.0in
\parskip=.07in
\parindent=0in
\pagestyle{fancy}

\newcommand{\setheader}[6]{
	\lhead{{\sc #1}\\{\sc #2} ({\small \it \today})}
	\rhead{
		{\bf #3} 
		\ifthenelse{\equal{#4}{}}{}{(#4)}\\
		{\bf #5} 
		\ifthenelse{\equal{#6}{}}{}{(#6)}%
	}
}

\newcommand{\bbox}[1]{
    \noindent\fbox{%
        \parbox{\textwidth}{%
            #1
        }%
    }
}

%%%
% Set up some shortcut commands
%%%
\newcommand{\R}{\mathbb{R}}
\newcommand{\N}{\mathbb{N}}
\newcommand{\Z}{\mathbb{Z}}
\newcommand{\Proj}{\mathrm{proj}}
\newcommand{\Perp}{\mathrm{perp}}
\newcommand{\proj}{\mathrm{proj}}
\newcommand{\Span}{\mathrm{span}}
\newcommand{\Null}{\mathrm{null}}
\newcommand{\Rank}{\mathrm{rank}}
\newcommand{\mat}[1]{\begin{pmatrix}#1\end{pmatrix}}

\begin{document}

\setheader{CSC165}{Lecture 2}{Arthur Gao}{}{}{}

Note: Problem Set 0 Available

\section{Functions:}
\boxed{f: A \to B \text{ means } f \text{ is a function from set } A \text{ to set } B}
\\
This means for each element of a ($a \in A)$ there's a corresponding $f(a) \in B$
\\
In the above definition, $A$ is the \textbf{domain} where $B$ is the \textbf{codomain} so we \textbf{MAP} elements from the domain to the codomain
\\
Example:\\
Take the sets:
\begin{align*}
	A = \{0, 2, 4\}\\
	B = \{1, 2, 3\}\\
\end{align*}
Define $f: A \to B$ by $f(x) = \frac{x}{2} + 1$ \\ \\
Some notes about Functions:
\begin{itemize}
	\item You cannot have one value in $A$ corresponding to multiple values of $B$
	\item You CAN have multiple values in $A$ corresponding to a single value of $B$
	\item We also say that $f(x)$ is called the image of $x$
	\begin{itemize}
		\item $f(x) \in B$. It is \textbf{NOT} the function since we do not know what $x$ is. 
		\item We must distinguish between the function $f$ and the element of $B$, $f(x)$
	\end{itemize}
	\item Say we define $f$ by $ f(x) = \frac{x^2-3x}{x-3}$ \dots well for what $f$
	\begin{itemize}
		\item We must define a domain
		\item Possible example could be $f: \R - {3} \to \R$
		\item Another option is "Define $f: \R \to \R \text{ by } g(x)$" which means "for each $x \in \R$, let $g(x) = x$"
		\item We must be clear about "what $x$" when we define functions
	\end{itemize}
\end{itemize}

\section{Predicates:}
Given: $P: A \to \{\text{True, False}\}$ so P maps to the set of boolean values\\
Example:\\
Define: $P: \R \to \{\text{T, F}\}$ by $P(x) : x > 165$
With this we can say the set of values in the domain where $P(x) = T$ which is:
$$\{x \in \R : P(x) = T\}$$
So we're interested in the "Set in the predicate where the given condition is true"
\\
\\
\section{Notation:}
$\sum$ is for adding up numbers where there is a sequential pattern.\\
Take: $$4 + \frac{9}{2} + \frac{16}{3} + \dots + \frac{164^2}{164}$$ 
which can be notated as $$\sum_{i = 1}^{164} \frac{(n + 1)^2}{n}$$ or $$\sum_{i = 2}^{165} \frac{n^2}{n - 1}$$
\textit{Note: A method for breaking down the above summation is puting it in a uniform form so rewrite it as:} \\
$$\frac{2^2}{1} + \frac{3^2}{2} + \frac{4^2}{3} + \dots + \frac{165^2}{164}$$
There is a convention that if we take a sum from on value to a value that is one fewer than the originating value, the sum is 0.\\
Example:
$$\sum_{i = 165}^{164} \frac{i}{i-1} = 0$$

Similarily, $\Pi$ is $f(a) * f(a + 1) * \dots * f(b)$ so:

\section{Propositional Logic}
\boxed{\text{A proposition is a statement that is true ... or it's false}} \\ \\
$x < 165$ is \textbf{not} a proposition since it is dependent on $x$\\
Some Propositional statements could be:
\begin{itemize}
	\item (valid) It's sunny in Los Angeles right now
	\item (valid) There is life on Europa
	\item (invalid) She likes cauliflower - well who is "she"?
\end{itemize}
\end{document}